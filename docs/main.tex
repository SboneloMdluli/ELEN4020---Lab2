%%%%%%%%%%%%%%%%%%%%%%%%%%%%%%%%%%%%%%%%%%%%%%%%%%%%%%%%%%%%%%%%%
\documentclass[10pt,journal]{article}
%%%%%%%%%%%%%%%%%%%%%%%%%%%%%%%%%%%%%%%%%
% Wenneker Assignment
% Structure Specification File
% Version 2.0 (12/1/2019)
%
% This template originates from:
% http://www.LaTeXTemplates.com
%
% Authors:
% Vel (vel@LaTeXTemplates.com)
% Frits Wenneker
%
% License:
% CC BY-NC-SA 3.0 (http://creativecommons.org/licenses/by-nc-sa/3.0/)
% 
%%%%%%%%%%%%%%%%%%%%%%%%%%%%%%%%%%%%%%%%%

%----------------------------------------------------------------------------------------
%	PACKAGES AND OTHER DOCUMENT CONFIGURATIONS
%----------------------------------------------------------------------------------------

\usepackage{amsmath, amsfonts, amsthm} % Math packages

\usepackage{listings} % Code listings, with syntax highlighting

\usepackage[english]{babel} % English language hyphenation

\usepackage{graphicx} % Required for inserting images
\graphicspath{{Figures/}{./}} % Specifies where to look for included images (trailing slash required)

\usepackage{booktabs} % Required for better horizontal rules in tables

\numberwithin{equation}{section} % Number equations within sections (i.e. 1.1, 1.2, 2.1, 2.2 instead of 1, 2, 3, 4)
\numberwithin{figure}{section} % Number figures within sections (i.e. 1.1, 1.2, 2.1, 2.2 instead of 1, 2, 3, 4)
\numberwithin{table}{section} % Number tables within sections (i.e. 1.1, 1.2, 2.1, 2.2 instead of 1, 2, 3, 4)

\setlength\parindent{0pt} % Removes all indentation from paragraphs

\usepackage{enumitem} % Required for list customisation
\setlist{noitemsep} % No spacing between list items

%----------------------------------------------------------------------------------------
%	DOCUMENT MARGINS
%----------------------------------------------------------------------------------------

\usepackage{geometry} % Required for adjusting page dimensions and margins

\geometry{
	paper=a4paper, % Paper size, change to letterpaper for US letter size
	top=2.5cm, % Top margin
	bottom=3cm, % Bottom margin
	left=3cm, % Left margin
	right=3cm, % Right margin
	headheight=0.75cm, % Header height
	footskip=1.5cm, % Space from the bottom margin to the baseline of the footer
	headsep=0.75cm, % Space from the top margin to the baseline of the header
	%showframe, % Uncomment to show how the type block is set on the page
}

%----------------------------------------------------------------------------------------
%	FONTS
%----------------------------------------------------------------------------------------

\usepackage[utf8]{inputenc} % Required for inputting international characters
\usepackage[T1]{fontenc} % Use 8-bit encoding

\usepackage{fourier} % Use the Adobe Utopia font for the document

%----------------------------------------------------------------------------------------
%	SECTION TITLES
%----------------------------------------------------------------------------------------

\usepackage{sectsty} % Allows customising section commands

\sectionfont{\vspace{6pt}\centering\normalfont\scshape} % \section{} styling
\subsectionfont{\normalfont\bfseries} % \subsection{} styling
\subsubsectionfont{\normalfont\itshape} % \subsubsection{} styling
\paragraphfont{\normalfont\scshape} % \paragraph{} styling

%----------------------------------------------------------------------------------------
%	HEADERS AND FOOTERS
%----------------------------------------------------------------------------------------

\usepackage{scrlayer-scrpage} % Required for customising headers and footers

\ohead*{} % Right header
\ihead*{} % Left header
\chead*{} % Centre header

\ofoot*{} % Right footer
\ifoot*{} % Left footer
\cfoot*{\pagemark} % Centre footer
 
\usepackage[ruled,vlined]{algorithm2e}
\usepackage{multirow}
\title{	
	\normalfont\largesize
	\textbf{\textsc{University of the Witwatersrand, Johannesburg}}\\
	\textsc{School of Electrical and Information Engineering}\\ 
	\vspace{5pt} % Whitespace
	\rule{\linewidth}{0.5pt}\\ % Thin top horizontal rule
	\vspace{10pt} % Whitespace
	{\huge ELEN4020A: Data Intensive Computing: Lab 2}\\ % The assignment title
	\vspace{1pt} % Whitespace
	\rule{\linewidth}{2pt}\\ % Thick bottom horizontal rule
	\vspace{1pt} % Whitespace
}

\author{ Sbonelo Mdluli(1101772), Heemal Ryan(792656), Haroon Rehman(1438756) } 

\date{\large\today}

\begin{document}
\maketitle 

\section{Introduction}
Matrix transposition is a key operation in various scientific and mathematical applications. Matrix transposition converts a M-rows-by-N columns array to a N-rows-by-M columns array. This lab introduces students to parallel programming through the use of Pthread and OpenMP libraries. Matrix transposition of large matrices can be a time and space expensive operation. Parallelism is used to improve the performance of the operation and to improve space requirement matrix transposition is to be done in place. This report will discuss the different matrix transposition algorithms that are developed with their performance for a 2D matrix with varying dimensions. 

\section{Matrix Transpose Algorithms}

The generated matrix is stored using row major ordering in a 1-D array with the exception of the block algorithm which uses C++ vectors to store the 2-D matrix. The use of vectors allows for a more efficient memory usage and most importantly to extract sub-matrices from the main matrix with ease. This is especially useful for block-oriented transposition.   

\subsection{Basic}

The basic matrix transpose algorithm sections the matrix into a lower and an upper triangle about the principal diagonal. The principal diagonal is is never transposed as it remains unchanged even when transposed. The algorithm uses two nested for loops, the outer and inner for loops specify an element in the upper triangle and the corresponding element in the lower triangle is determined based on the index of the element is the upper matrix.

\subsection{Block}

Initially recursion used to perform the block transpose algorithm however this implementation was later changed as recursion is memory intensive and defeated the purpose of the lab. The algorithm used here entails a twofold process. Firstly, contiguous square sub-matrices of certain length from the main matrix are extracted. These sub-matrices are transposed independently, and placed back into the main matrix. Secondly, after all the sub-matrices are transposed, the main matrix is transposed "block" wise as normal matrix transposition. These blocks are of the same length as the sub-matrices mentioned in part 1. 

\subsection{Diagonal}

The diagonal algorithms uses the basic matrix transpose for its functionality. This algorithm also makes uses of two nested for loops, however the outer for loop starts at a specified diagonal. The diagonal is not specified using the array index but can be accessed numerically with the top left diagonal equal to zero. The transposition only takes place at the diagonal where the row and column intersect. 

\section{Pthreads}

POSIX threads are used to achieve parallelism in shared memory. Pthreads provide the programmer with a lot of freedom and allows one to produce portable multithreded code. 

\subsection{Diagonal-Threading}
The algorithm makes use of 8 Threads which were determined to be the most efficient when compared to other variations. Initially all 8 threads will be assigned to work on specified diagonal corresponding to its thread id. Once a thread has finished transposing it increments a global variable \textit {next\_pos} which is used to communicate the next available diagonal. To avoid race conditions \textit {pthread\_mutex\_lock} is used to only allowed one thread to update this variable at a time. The threads terminate once \textit {next\_pos} is equal to the size of the matrix. Structs were used to better represent the information the threads needed to work properly. 

\newline 

\subsection{Block-Oriented}

\section{OpenMP}

Open specifications for Multi Processing (OpenMP) offers a high level shared memory parallelism model compared to pthreads. OpenMP is based on the fork-join execution model that is at the beginning of a program the master thread executes and forks whenever there is a parallel block. OpenMP is often easier to use compared to pthreads since the compiler does most of the low level work.

\subsection{Naive}
The naive implementation of the matrix transposition algorithm is heavily based on the basic algorithm. To parallelize the nested for-loops two compiler directives are used namely :
\newline
{\#pragma omp parallel shared(arr) private( i, j) num_threads(NUM\_THREADS) } 
\newline
{\#pragma omp for schedule(static,size)  nowait}
\newline

The first directive parallelizes the nested loops, variable \textit{i}, \textit{j} are made private for each thread to ensure that the threads perform independent transposition. We use schedule because we have a work sharing for-loop. Nowait is used to instruct the thread to continue computing the next transpose

\subsection{Diagnoal-Threading}
The diagonal version of the version in OpenMP shares great similarities with that naive implementation with the exception that scheduling is dynamic and each thread works on a chunk of size one. The scheduling was changed to dynamic as the static one is less efficient and dynamic scheduling allowed a thread to move on to the next task once it finished its execution.

\subsection{Block-Oriented}
In this lab, tensors are restricted to $n$ dimensional square tensors which simplifies some aspects of the computation. The An empty  $2\times 2$ matrix of size equal to the one of the matrices being added is created. The resultant values are stored in the empty matrix. Equation describes tensor multiplication of  $2\times 2$ matrices and how each element is generated in the resultant matrix.


%%%%%%%%%%%%%%%%%%%%%%%%%%%%%%%%%%%%%%%%%%%%%%%%%%%%%%%%%%%%%%%%%
\section{Results}\
% Please add the following required packages to your document preamble:

\begin{table}[]
\caption{\label{tab:table-name}Algorithm Benchmark running time}
\begin{tabular}{|c|c|c|c|c|c|c|}

\hline

\multirow{2}{*}{\textbf{$N_{0}=N_{1}$}} & \textbf{Basic} & \multicolumn{2}{c|}{\textbf{Pthreads}} & \multicolumn{3}{c|}{\textbf{OpenMP}} \\ \cline{2-7} 
                                    &                & Diagonal           & Blocked           & Naive       & Diagonal            & Blocked           \\ \hline
\textbf{128}                        & 0.156 ms       & 0.613 ms           & 8.889 ms          &             & 3.253 ms            & 9.615 ms          \\ \hline
\textbf{1024}                       & 4.471 ms       & 2.165 ms           & 0.345 s          &             & 16.84 ms            & 0.350 s          \\ \hline
\textbf{2048}                       & 37.70 ms       & 8.457 ms           & 1.387 s           &             & 76.67 ms            & 1.398 s           \\ \hline
\textbf{4096}                       & 0.233 s        & 44.62ms            & 5.415 s           &             & 0.321 s             & 5.465 s           \\ \hline
\end{tabular}
\end{table}

\section{Conclusion}\

\justify
One can further verify the results using numpy for the rank2Tensor procedures. From the algorithm analysis it is evident that computationally complexity increases with the rank and dimensions of the tensor. The procedures developed in this lab are presented for $n\times n$ or $n\times n\times n$ however modifications may be made to make them more flexible to work in varying dimensions.



%%%%%%%%%%%%%%%%%%%%%%%%%%%%%%%%%%%%%%%%%%%%%%%%%%%%%%%%%%%%%%
\appendix
\newpage
\section{Appendix}
\subsection{General Function Used Across Algorithms}
%%%%%%%%%%%%%%%%%%%%%%%%%%%%%%%%%%%%%%%%%%%%%%%%%%%%%%%%%%%%%%
\begin{algorithm}[H]
\SetAlgoLined
\textbf{Function name:} \emph{Swap}\\
Input arguments:\: int* num1, int* num2\\

\vspace{10pt} 
Initialization\:long int temp = 0\\
     temp = *num\\
    *num1 = *num2\\
    *num2 = temp\\

\caption{Void Swap Function}
\end{algorithm}


%%%%%%%%%%%%%%%%%%%%%%%%%%%%%%%%%%%%%%%%%%%%%%%%%%%%%%%%%%%%%%
\subsection{Naive Approach}
\begin{algorithm}[H]
\SetAlgoLined
\textbf{Function name:} \emph{*CreateMatrix}\\
Input arguments:\:matrixSize\\

Initialization\::\\ 
 int *arr = (int *)calloc(matrixSize * matrixSize, sizeof(int))\\
 val = 1
 
\vspace{10pt} 
\For{i=0  to size-1}{
    \For{j= 0 to size-1}{*(arr + i*size + j) = val++\\
}}

\caption{Void Function to Create and Populate Input Matrix}
\end{algorithm}

%%%%%%%%%%%%%%%%%%%%%%%%%%%%%%%%%%%%%%%%%%%%%%%%%%%%%%%%%%%%%%
\begin{algorithm}[H]
\SetAlgoLined
\textbf{Function name:} \emph{Transpose}\\
Input arguments:\: int* arr, int matrixSize \\

\vspace{10pt} 
\For{i=0  to size-1}{
    \For{j= 0 to size-1}{swap( (arr + i*matrixSize +j), (arr + j*matrixSize +i)\\
}}

\caption{Void Transpose Function}
\end{algorithm}
%%%%%%%%%%%%%%%%%%%%%%%%%%%%%%%%%%%%%%%%%%%%%%%%%%%%%%%%%%%%%%

\newpage
%%%%%%%%%%%%%%%%%%%%%%%%%%%%%%%%%%%%%%%%%%%%%%%%%%%%%%%%%%%%%%
\subsection{Diagonal Approach - PThreads}
%%%%%%%%%%%%%%%%%%%%%%%%%%%%%%%%%%%%%%%%%%%%%%%%%%%%%%%%%%%%%%
\textbf{Global Variables:} \\
int NumberOfThreads = 8\\
int nextPosition = 1\\
int *arr\\

%%%%%%%%%%%%%%%%%%%%%%%%%%%%%%%%%%%%%%%%%%%%%%%%%%%%%%%%%%%%%%
\begin{algorithm}[H]
\SetAlgoLined
\textbf{Function name:} \emph{*CreateMatrix}\\
Initialization\::\\
int *C = (int *)calloc(matrixSize * matrixSize, sizeof(int))\\
val = 1

\vspace{10pt} 
\For{i=0  to size-1}{
    \For{j= 0 to size-1}{ *(C + i*matrixSize + j) = val++
}}
\Return C
\caption{int Function to Create and Populate Input Matrix}
\end{algorithm}

%%%%%%%%%%%%%%%%%%%%%%%%%%%%%%%%%%%%%%%%%%%%%%%%%%%%%%%%%%%%%%
\begin{algorithm}[H]
\SetAlgoLined
\textbf{Function name:} \emph{*Transpose}\\
Initialization\::\\
struct details* local = args\\
pos= 0, n= 0, nextPosition= 1\\

\vspace{10pt} 
\While{pos < matrixSize}{pos = local->index\\
    \For{i=pos  to matrixSize-1}{
        \For{j = pos+1 to matrixSize-1}{swap((arr + pos*matrixSize + j),(arr + j*matrixSize + pos))\\
        }break
    }
    \textbf{pthreadmMutexLock(\&lock)}\\
	n = nextPosition++\\
	local->index = n\\
	\textbf{pthreadMutexUnlock(\&lock)}\\
    
}
\caption{Void Transpose Function}
\end{algorithm}

%%%%%%%%%%%%%%%%%%%%%%%%%%%%%%%%%%%%%%%%%%%%%%%%%%%%%%%%%%%%%%
\begin{algorithm}[H]
\SetAlgoLined
\textbf{Function name:} \emph{Main Function}\\
Initialization\::\\
matrixSize={128, 1024, 2048 ,4096}\\
struct details args[NumberOfThreads]\\
pthread-t threads[matrixSize]\\

\vspace{3pt} 
\textbf{Call to Create\::}\\
arr = CreateMatrix(matrixSize)\\

\vspace{10pt} 
\For{i=0  to NumberOfThreads}{
     args[i].index = i\\
	 \textbf{pthread-create(\&threads[i],NULL, transpose,(void*) \&args)}\\
}

\vspace{10pt} 
\For{i=0  to NumberOfThreads}{
     args[i].index = i\\
	 \textbf{pthread-join(threads[i],NULL)}\\
}
exit(0)

\caption{Main Function - PThreads}
\end{algorithm}
%%%%%%%%%%%%%%%%%%%%%%%%%%%%%%%%%%%%%%%%%%%%%%%%%%%%%%%%%%%%%%

\newpage
%%%%%%%%%%%%%%%%%%%%%%%%%%%%%%%%%%%%%%%%%%%%%%%%%%%%%%%%%%%%%%
\subsection{Diagonal Approach - OpenMP}
%%%%%%%%%%%%%%%%%%%%%%%%%%%%%%%%%%%%%%%%%%%%%%%%%%%%%%%%%%%%%%
\textbf{Global Variables:} \\
int NumberOfThreads = 8\\

%%%%%%%%%%%%%%%%%%%%%%%%%%%%%%%%%%%%%%%%%%%%%%%%%%%%%%%%%%%%%%
\begin{algorithm}[H]
\SetAlgoLined
\textbf{Function name:} \emph{*CreateMatrix}\\
Input arguments:\:matrixSize\\
Initialization\::\\
int *arr = (int *)calloc(size * size, sizeof(int))\\
val = 1;

\vspace{10pt} 
\textbf{pragma omp for schedule (static, size)}\\
\For{i=0  to size-1}{
    \For{j= 0 to size-1}{*(arr + i*size + j) = val++
}}
\Return arr
\caption{int Function to Create and Populate Input Matrix}
\end{algorithm}

%%%%%%%%%%%%%%%%%%%%%%%%%%%%%%%%%%%%%%%%%%%%%%%%%%%%%%%%%%%%%%
\begin{algorithm}[H]
\SetAlgoLined
\textbf{Function name:} \emph{*Transpose}\\
Input arguments:\:matrixSize, int*arr\\
Initialization\::pos= 0,\\

\vspace{10pt} 
\While{pos < matrixSize}{pos = local->index\\
\textbf{pragma omp parallel shared(pos) private( i, j) num-threads(NumberOfThreads}\\ 
    \For{i=pos  to matrixSize-1}{
        \For{j = pos+1 to matrixSize-1}{swap((arr + pos*matrixSize + j),(arr + j*matrixSize + pos))\\
        }break
    }
    \textbf{pragma omp critical}{pos++;}

}
\caption{Void Transpose Function}
\end{algorithm}

%%%%%%%%%%%%%%%%%%%%%%%%%%%%%%%%%%%%%%%%%%%%%%%%%%%%%%%%%%%%%%
\begin{algorithm}[H]
\SetAlgoLined
\textbf{Function name:} \emph{Main Function}\\
Initialization\::\\
matrixSize={128, 1024, 2048 ,4096}\\

\vspace{10pt} 
\For{i=0  to 3}{n = *(size +i)\\ 
                int *A = CreateMatrix(n)\\
                transpose(A,n)\\
}
 exit(0)
\caption{Main Function - OpenMP}
\end{algorithm}


%%%%%%%%%%%%%%%%%%%%%%%%%%%%%%%%%%%%%%%%%%%%%%%%%%%%%%%%%%%%%%
\subsection{Commonly Used Functions in Block Algorithm}
\begin{algorithm}[H]
\SetAlgoLined
\textbf{Function name:} \emph{void createMatrix}\\
Initialization\::long int val=1, vector<long int> num\\

\vspace{10pt} 
\For{i=0  to matrixSize-1}{
    \For{j= 0 to matrixSize-1}{
            num.pushBack(val)\\
            val++\\
}MATRIX.pushBack(num);
           num.clear();}
\caption{Void Function to Create and Populate Input Matrix}
\end{algorithm}

%%%%%%%%%%%%%%%%%%%%%%%%%%%%%%%%%%%%%%%%%%%%%%%%%%%%%%%%%%%%%%
\begin{algorithm}[H]
\SetAlgoLined
\textbf{Function name:} \emph{tensor getSubMatrix}\\
Input arguments:\: int A, int B\\
Initialization\::\\tensor sub(block-len,vector<long int>(block-len))\\

\vspace{10pt} 
\For{i=0  to block-length}{
    \For{j= 0 to block-length}{sub[i][j] = MATRIX[A+i][B+j]\\
}}
\caption{Void Function to get SubMatrix}
\end{algorithm}

%%%%%%%%%%%%%%%%%%%%%%%%%%%%%%%%%%%%%%%%%%%%%%%%%%%%%%%%%%%%%%
\begin{algorithm}[H]
\SetAlgoLined
\textbf{Function name:} \emph{tensor setSubMatrix}\\
Input arguments:\: int A, int B, tensor Block\\

\vspace{10pt} 
\For{i=0  to block-length}{
    \For{j= 0 to block-length}{MATRIX[A+i][B+j]=block[i][j]\\
}}
\caption{Void Function to set SubMatrix}
\end{algorithm}

%%%%%%%%%%%%%%%%%%%%%%%%%%%%%%%%%%%%%%%%%%%%%%%%%%%%%%%%%%%%%%
\begin{algorithm}[H]
\SetAlgoLined
\textbf{Function name:} \emph{void transpose}\\
Input arguments:\: int matrixSize\\
Initialization\:: size = arr.size()\\

\vspace{10pt} 
\For{i=0  to matrixSize-1}{
    \For{j= 0 to matrixSize-1}{swap(arr[i][j],arr[j][i])\\
}}
\caption{Void Function to perform normal 2D x 2D Transformations}
\end{algorithm}

\subsection{Block Approach - OpenMP}
%%%%%%%%%%%%%%%%%%%%%%%%%%%%%%%%%%%%%%%%%%%%%%%%%%%%%%%%%%%%%%
\textbf{Global Variables:} \\

using tensor = vector <vector<long int>>\\
int block-len=2\\
long int size-mat = 8\\
tensor MATRIX\\
const int NumberOfThreads=8\\
int nextPosition=NumberOfThreads\\


%%%%%%%%%%%%%%%%%%%%%%%%%%%%%%%%%%%%%%%%%%%%%%%%%%%%%%%%%%%%%%
\begin{algorithm}[H]
\SetAlgoLined
\textbf{Function name:} \emph{void shuffle}\\

\vspace{10pt} 
    \textbf{pragma omp parallel num-threads(NUM-THREADS)}

\For{i=0  to matrixSize-1}{i+=block-length\\
    \textbf{pragma omp for schedule(static,size-mat) nowait}
    \For{j= 0 to matrixSize-1}{j+=block-length\\
            tensor block = getSubMatrix(i,j)\\
            tensor block2 = getSubMatrix(j,i)\\
            setSubMatrix(i,j,block2)\\
            setSubMatrix(j,i,block)\\
}}
\caption{Void Function to Transpose actual bigger blocks}
\end{algorithm}

%%%%%%%%%%%%%%%%%%%%%%%%%%%%%%%%%%%%%%%%%%%%%%%%%%%%%%%%%%%%%%
\begin{algorithm}[H]
\SetAlgoLined
\textbf{Function name:} \emph{*void transpose-blocks}\\
Initialization\::int thread-counter = 0\\
    long local = thread-counter\\
    thread-counter++, int pos = 0\\
    int posY = 0 , int tmp = size-mat/block-len\\
    int x = 0 , tensor block\\

\vspace{10pt} 
\textbf{pragma omp parallel shared (thread-counter) num-threads(NumberOfThreads)}\\
\While{1}{
        pos = (block-len*local)\\
        x = local/tmp\\
        posY = x*block-len\\
        \eIf {posY >= size-mat}{break}{\\
        block = getSubMatrix(posY,pos)\\
        transpose(block)\\
        setSubMatrix(posY,pos,block)\\

        \textbf{pragma omp critical}\\
            local= next-pos\\
            next-pos++
	  }
}
\caption{Void Transpose Function: Transposes all inner blocks of matrix}
\end{algorithm}

%%%%%%%%%%%%%%%%%%%%%%%%%%%%%%%%%%%%%%%%%%%%%%%%%%%%%%%%%%%%%%
\begin{algorithm}[H]
\SetAlgoLined
\textbf{Function name:} \emph{Main Function}\\
\textbf{Call to Create\::}\\
CreateMatrix()\\
transpose-blocks()\\
shuffle()\\
\Return(0)\\

\caption{Main Function - OpenMP}
\end{algorithm}

%%%%%%%%%%%%%%%%%%%%%%%%%%%%%%%%%%%%%%%%%%%%%%%%%%%%%%%%%%%%%%

\newpage
%%%%%%%%%%%%%%%%%%%%%%%%%%%%%%%%%%%%%%%%%%%%%%%%%%%%%%%%%%%%%%
\subsection{Block Approach - PThreads}
\textbf{Global Variables:} \\
int block-len=2\\
long int size-mat =8\\
tensor MATRIX\\
const int NUM-THREADS=8\\
int next-pos=NUM-THREADS\\
pthread-mutex-t lock\\

%%%%%%%%%%%%%%%%%%%%%%%%%%%%%%%%%%%%%%%%%%%%%%%%%%%%%%%%%%%%%%
\begin{algorithm}[H]
\SetAlgoLined
\textbf{Function name:} \emph{void shuffle}\\

\vspace{10pt} 
\For{i=0  to matrixSize-1}{i+=block-length\\
    \For{j= 0 to matrixSize-1}{j+=block-length\\
            tensor block = getSubMatrix(i,j)\\
            tensor block2 = getSubMatrix(j,i)\\
            setSubMatrix(i,j,block2)\\
            setSubMatrix(j,i,block)\\
}}
\caption{Void Function to Transpose actual bigger blocks}
\end{algorithm}

%%%%%%%%%%%%%%%%%%%%%%%%%%%%%%%%%%%%%%%%%%%%%%%%%%%%%%%%%%%%%%
\begin{algorithm}[H]
\SetAlgoLined
\textbf{Function name:} \emph{*void transpose-blocks}\\
Initialization\::int thread-counter = 0\\
    long local = thread-counter\\
    thread-counter++, int pos = 0\\
    int posY = 0 , int tmp = size-mat/block-len\\
    int x = 0 , tensor block\\

\vspace{10pt} 

\While{1}{
        pos = (block-len*local)\\
        x = local/tmp\\
        posY = x*block-len\\
        \eIf {posY >= size-mat}{break}{\\
        block = getSubMatrix(posY,pos)\\
        transpose(block)\\
        setSubMatrix(posY,pos,block)\\

        \textbf{pthread-mutex-lock(\&lock)}\\
            local= next-pos\\
            next-pos++\\
        \textbf{pthread-mutex-unlock(&lock)}
	  }
}
\caption{Void Transpose Function: Transposes all inner blocks of matrix}
\end{algorithm}

%%%%%%%%%%%%%%%%%%%%%%%%%%%%%%%%%%%%%%%%%%%%%%%%%%%%%%%%%%%%%%
\begin{algorithm}[H]
\SetAlgoLined
\textbf{Function name:} \emph{Main Function}\\
\textbf{Call to Create\::}\\
CreateMatrix()\\
Initialization\::int index[NUM-THREADS]\\

\textbf{pthread-t threads[NUM-THREADS]}

\vspace{10pt} 
\For{i=0  to NUM-THREADS-1}{
index[i]=i\\
pthread-create(&threads[i],NULL,transpose-blocks,(void*)index[i])\\
}
\For{i=0  to NUM-THREADS-1}{
pthread-join(threads[i],NULL)\\
}
\Return(0)\\

\caption{Main Function - PThreads}
\end{algorithm}
%%%%%%%%%%%%%%%%%%%%%%%%%%%%%%%%%%%%%%%%%%%%%%%%%%%%%%%%%%%%%%%%%


\end{document}
